\documentclass[a4paper,10pt]{article}

\usepackage[utf8]{inputenc}
\usepackage{geometry}
\geometry{margin=1in}
\usepackage{titlesec}
\usepackage{enumitem}
\usepackage{hyperref}
\usepackage[dvipsnames]{xcolor}
\usepackage{tabularx}
\usepackage{fancyhdr}

% Kolory
\definecolor{HeaderBlue}{RGB}{0, 51, 102}
\definecolor{LinkBlue}{RGB}{0, 102, 204}

\usepackage{fontspec}
\setmainfont{Roboto}
\setsansfont{Montserrat}

% Formatowanie nagłówków
\titleformat{\section}{\Large\bfseries\color{HeaderBlue}}{}{0em}{}[\titlerule]
\titleformat{\subsection}{\large\bfseries}{}{0em}{}
\titlespacing{\section}{0pt}{16pt}{8pt}
\titlespacing{\subsection}{0pt}{10pt}{4pt}

% Linki
\hypersetup{
    colorlinks=true,
    linkcolor=LinkBlue,
    filecolor=LinkBlue,
    urlcolor=LinkBlue,
}

% Stopka
\pagestyle{fancy}
\fancyhead{}
\fancyfoot{}
\fancyfoot[C]{\footnotesize Wyrażam zgodę na przetwarzanie moich danych osobowych zawartych w mojej aplikacji na potrzeby procesu rekrutacji, zgodnie z ustawą z dnia 10 maja 2018 r. o ochronie danych osobowych oraz zgodnie z Rozporządzeniem Parlamentu Europejskiego i Rady (UE) 2016/679 z dnia 27 kwietnia 2016 r.}
\renewcommand{\headrulewidth}{0pt}

\begin{document}

% Nagłówek
\begin{center}
    {\Huge\textbf{Sebastian Sadowy}} \\[0.3cm]
    \href{https://sebastiansadowy.com}{sebastiansadowy.com} \\[0.2cm]
    \begin{tabular}{c c c}
        \href{mailto:sebastian.sadowy@proton.me}{sebastian.sadowy@proton.me} &
        \textbf{·} &
        +48 504 602 420 \\
        \href{https://github.com/zdsdd}{github.com/zdsdd} &
        \textbf{·} &
        \href{https://www.linkedin.com/in/sebastiansadowy}{linkedin.com/in/sebastiansadowy}
    \end{tabular} \\
    Łódź / Zdalnie
\end{center}

% O mnie
\section{O mnie}

Interesuję się zastosowaniem sztucznej inteligencji w automatyzacji procesów oraz budowaniem rozwiązań opartych o chmurę i praktyki DevOps. Cenię prostotę w projektowaniu systemów i jasną komunikację w zespole, choć równie dobrze odnajduję się w samodzielnej pracy.

% Doświadczenie zawodowe
\section{Doświadczenie zawodowe}

\subsection{Stażysta ds. Azure i GenAI | Transition Technologies PSC}
\textit{Marzec 2025--obecnie}
\begin{itemize}[leftmargin=0.5cm, nosep]
    \item Wykorzystywanie \textbf{Azure AI Foundry}, \textbf{Python} i \textbf{TypeScript} do tworzenia rozwiązań opartych na sztucznej inteligencji.
    \item Tworzenie infrastruktury jako kod (IaC) oraz integracja procesów CI/CD przy użyciu \textbf{GitHub Actions}.
    \item Zarządzanie infrastrukturą w GitHub i Azure z wykorzystaniem \textbf{Terraform}.
\end{itemize}

\subsection{Stażysta – Programista | Transition Technologies PSC}
\textit{Lipiec 2024--Sierpień 2024}
\begin{itemize}[leftmargin=0.5cm, nosep]
    \item Projektowanie potoków trenowania modeli AI przy użyciu \textbf{TensorFlow}.
    \item Tworzenie interfejsu użytkownika w \textbf{React} oraz backendu w \textbf{Flask}.
\end{itemize}

\subsection{Stażysta – Twórca Gier | Cleversan Games}
\textit{Kwiecień 2024--Maj 2024}
\begin{itemize}[leftmargin=0.5cm, nosep]
    \item Programowanie mechanik gry oraz optymalizacja wydajności w \textbf{C\#} i \textbf{Unity}.
    \item Profilowanie kodu i implementacja optymalizacji w silniku Unity.
\end{itemize}

% Umiejętności techniczne
\section{Umiejętności techniczne}

\begin{itemize}[leftmargin=0.5cm, itemsep=6pt, label=\textbullet]
    \item \textbf{Języki programowania:} C\#, Go, Python, Java, C/C++, TypeScript, JavaScript, Bash
    \item \textbf{Frameworki i biblioteki:} .NET, React, Vue, Node.js, Tailwind, gRPC, Flask, TensorFlow, Spring
    \item \textbf{Chmura:} AWS (S3, Lambda, EC2, RDS, Amplify, ECS, ECR), Azure (Functions, IoT Hub)
    \item \textbf{Systemy kolejkowe:} RabbitMQ, AWS SQS, Azure Service Bus
    \item \textbf{Bazy danych:} PostgreSQL, Cosmos DB, Redis
    \item \textbf{DevOps i CI/CD:} Docker, Kubernetes, Terraform, GitHub Actions
    \item \textbf{Tworzenie API:} REST, OpenAPI, Swagger
    \item \textbf{Narzędzia:} Visual Studio, VS Code, JetBrains, Postman, WSL
    \item \textbf{Kontrola wersji:} Git, GitHub, GitLab
    \item \textbf{Dokumentacja:} UML, Docusaurus
    \item \textbf{Inżynieria oprogramowania:} TDD, SOLID, mikroserwisy, Event-Driven Architecture
    \item \textbf{Języki:} polski (ojczysty), angielski (B2)
\end{itemize}

% Projekty własne
\section{Projekty własne}

\subsection{Aplikacja do zarządzania konferencjami}
\textit{JavaScript (Vue, Express.js), Docker, PostgreSQL}
\begin{itemize}[leftmargin=0.5cm, nosep]
    \item Aplikacja full-stack do zarządzania konferencjami z aktualizacjami w czasie rzeczywistym.
    \item Implementacja uwierzytelniania i kategorii wydarzeń.
\end{itemize}

\subsection{Platforma e-commerce do sprzedaży zakładek}
\textit{Go, React, AWS, Docker}
\begin{itemize}[leftmargin=0.5cm, nosep]
    \item Skalowalna platforma full-stack do sprzedaży rękodzieła.
    \item Backend z obsługą uwierzytelniania i hostingu obrazów.
\end{itemize}

% Edukacja
\section{Edukacja}

\subsection{Magister Informatyki | Uniwersytet Łódzki}
\textit{Wydział Matematyki i Informatyki \hfill Październik 2024--obecnie}
\begin{itemize}[leftmargin=0.5cm, nosep]
  \item Specjalizacja: Informatyka ogólna
  \item Stypendium Rektora za wyniki w nauce (2024)
\end{itemize}

\subsection{Licencjat z Informatyki | Uniwersytet Łódzki}
\textit{Wydział Matematyki i Informatyki \hfill Październik 2021--Czerwiec 2024}
\begin{itemize}[leftmargin=0.5cm, nosep]
  \item Specjalizacja: Grafika i projektowanie gier
  \item Ocena końcowa: \textbf{bardzo dobra} (średnia: \textbf{4,77})
  \item Stypendium Rektora za wyniki w nauce (2022, 2023)
\end{itemize}

% Zainteresowania
\section{Zainteresowania}

\begin{itemize}[leftmargin=0.5cm, nosep]
    \item Systemy rozproszone, przetwarzanie w chmurze, mikroserwisy
    \item Kino, gry komputerowe, literatura
\end{itemize}

\end{document}
